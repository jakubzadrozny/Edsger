%%%%%%%%%%%%%%%%%%%%%%%%%%%%%%%%%%%%%%%%%
% Short Sectioned Assignment
% LaTeX Template
% Version 1.0 (5/5/12)
%
% This template has been downloaded from:
% http://www.LaTeXTemplates.com
%
% Original author:
% Frits Wenneker (http://www.howtotex.com)
%
% License:
% CC BY-NC-SA 3.0 (http://creativecommons.org/licenses/by-nc-sa/3.0/)
%
%%%%%%%%%%%%%%%%%%%%%%%%%%%%%%%%%%%%%%%%%

%----------------------------------------------------------------------------------------
%	PACKAGES AND OTHER DOCUMENT CONFIGURATIONS
%----------------------------------------------------------------------------------------

\documentclass[paper=a4, fontsize=11pt]{scrartcl} % A4 paper and 11pt font size

\usepackage[margin=2.5cm]{geometry}
\usepackage{fourier} % Use the Adobe Utopia font for the document - comment this line to return to the LaTeX default
\usepackage[utf8]{inputenc}
\usepackage[polish]{babel} % English language/hyphenation

\usepackage{sectsty} % Allows customizing section commands
\allsectionsfont{\normalfont\scshape} % Make all sections centered, the default font and small caps

\setlength{\parskip}{1em}

%----------------------------------------------------------------------------------------
%	TITLE SECTION
%----------------------------------------------------------------------------------------

\newcommand{\horrule}[1]{\rule{\linewidth}{#1}} % Create horizontal rule command with 1 argument of height

\title{
\normalfont \normalsize
\textsc{Instytut Informatyki Uniwersytetu Wrocławskiego} \\ [25pt] % Your university, school and/or department name(s)
\horrule{0.5pt} \\[0.4cm] % Thin top horizontal rule
\huge Edsger - prosty nawigator \\ % The assignment title
\horrule{0.5pt} \\[0.4cm] % Thick bottom horizontal rule
}

\author{Jakub Zadrożny} % Your name

\date{\normalsize\today} % Today's date or a custom date

\begin{document}
\pagenumbering{gobble}

\maketitle % Print the title

\section{Wymagania}
Do poprawnego skompilowania programu potrzebna jest biblioteka GTK+ w wersji 3.10 lub wyższej oraz narzędzie \texttt{make}.

\noindent Wspierane systemu operacyjne: macOS i Linux.
\section{Kompilacja}
Aby skompilować program, należy wydać polecenie \texttt{make} w katalogu głównym aplikacji.
\section{Instrukcja obsługi}
Aplikacja składa się z dwóch głównych części -- mapy sieci komunikacyjnej
(szary obszar po lewej stronie) oraz panelu sterowania (biały obszar po prawej stronie).

\noindent Poniżej znajdują się instrukcje wykonania operacji udostępnianych przez program:
\begin{enumerate}
    \item Dodawanie wierzchołka do mapy -- aby dodać nowy wierzchołek do mapy, należy wcisnąć przycisk ,,Add vertex''.
    \item Usuwanie wierzchołka z mapy -- aby usunąć wierzchołek z mapy, należy kliknąć go prawym przyciskiem myszy,
        po czym wybrać opcję ,,Destroy vertex'' z rozwijanego menu.
    \item Edycja etykiety wierzchołka -- aby zmienić etykietę wierzchołka, należy kliknąć go prawym przyciskiem myszy,
        po czym wybrać opcję ,,Edit label'' z rozwijanego menu. W nowo otwartym okienku należy wpisać nową etykietę, po czym
        kliknąć przycisk ,,OK''.
    \item Dodawanie krawędzi do mapy -- aby dodać nową krawędź skierowaną do mapy, należy kliknąć prawym przyciskiem myszy
        na wierzchołek startowy, po czym wybrać opcję ,,New edge'', przesunąć kursor nad wierzchołek docelowy i kliknąć go lewym
        przyciskiem myszy.
    \item Usuwanie krawędzi z mapy -- aby usunąć krawędź z mapy, należy kliknąć prawym przyciskiem w okolicy strzałki kierunkowej
        wybranej krawędzi, po czym wybrać opcję ,,Remove edge'' z rozwijanego menu.
    \item Zmiana wagi krawędzi -- aby zmienić wagę krawędzi, należy kliknąć prawym przyciskiem myszy w okolicy strzałki kierunkowej
        wybranej krawędzi, po czym wybrać opcję ,,Edit weight'' z rozwijanego menu. W nowo otwartym okienku należy podać nową wagę
        krawędzi i wcisnąć przycisk ,,OK''.
    \item Wyszukiwanie najkrótszej ścieżki pomiędzy zadanymi wierzchołkami -- aby znaleźć najkrótzą scieżkę pomiędzy dwoma wierzchołkami,
        należy wpisać (lub wybrać z rozwijanej listy) etykietę wierzchołka startowego oraz docelowego, po czym wcisnąć przycisk
        ,,Find route''.
    \item Zapisywanie stanu mapy do pliku -- aby zapisać aktualny stan mapy, należy użyć przycisku ,,Save map''.
    \item Ładowanie stanu mapy z pliku -- aby wczytać stan mapy z pliku, należy użyć przycisku ,,Load map''.
\end{enumerate}

\section{Struktura programu}
Aplikacja składa się z pliku głównego \texttt{main.c} oraz następujących czterech modułów:
\begin{itemize}
    \item \texttt{interface} -- odpowiedzialnego za obsługę interfejsu graficznego,
    \item \texttt{graph} -- odpowiedzialnego za przechowywanie grafu i wykonywanie na nim operacji,
    \item \texttt{list} -- implementującego zmodyfikowaną strukturę listy,
    \item \texttt{saving} -- odpowiedzialnego za zapisywanie oraz ładowanie mapy.
\end{itemize}

\noindent Dodatkowo w folderze znajduje się również plik \texttt{Makefile} odpowiedzialny za usprawnienie procesu kompilacji.

\end{document}
