%%%%%%%%%%%%%%%%%%%%%%%%%%%%%%%%%%%%%%%%%
% Short Sectioned Assignment
% LaTeX Template
% Version 1.0 (5/5/12)
%
% This template has been downloaded from:
% http://www.LaTeXTemplates.com
%
% Original author:
% Frits Wenneker (http://www.howtotex.com)
%
% License:
% CC BY-NC-SA 3.0 (http://creativecommons.org/licenses/by-nc-sa/3.0/)
%
%%%%%%%%%%%%%%%%%%%%%%%%%%%%%%%%%%%%%%%%%

%----------------------------------------------------------------------------------------
%	PACKAGES AND OTHER DOCUMENT CONFIGURATIONS
%----------------------------------------------------------------------------------------

\documentclass[paper=a4, fontsize=11pt]{scrartcl} % A4 paper and 11pt font size

\usepackage[margin=2.5cm]{geometry}
\usepackage{fourier} % Use the Adobe Utopia font for the document - comment this line to return to the LaTeX default
\usepackage[utf8]{inputenc}
\usepackage[polish]{babel} % English language/hyphenation

\usepackage{sectsty} % Allows customizing section commands
\allsectionsfont{\centering \normalfont\scshape} % Make all sections centered, the default font and small caps

\setlength{\parskip}{1em}

%----------------------------------------------------------------------------------------
%	TITLE SECTION
%----------------------------------------------------------------------------------------

\newcommand{\horrule}[1]{\rule{\linewidth}{#1}} % Create horizontal rule command with 1 argument of height

\title{	
\normalfont \normalsize 
\textsc{Instytut Informatyki UWr} \\ [25pt] % Your university, school and/or department name(s)
\horrule{0.5pt} \\[0.4cm] % Thin top horizontal rule
\huge Edsger - prosty nawigator \\ % The assignment title
\horrule{0.5pt} \\[0.4cm] % Thick bottom horizontal rule
}

\author{Jakub Zadrożny} % Your name

\date{\normalsize\today} % Today's date or a custom date

\begin{document}
\pagenumbering{gobble}

\maketitle % Print the title

\section*{Opis projektu}

Celem projektu Edsger jest stworzenie aplikacji służącej do nawigacji użytkownika po zadanej sieci komunikacyjnej. Podstawową funkcją programu będzie wytyczenie najkrótszej ścieżki pomiędzy wybranymi przez użytkownika wierzchołkami oraz zaprezentowanie znalezionego rozwiązania.

Sieć komunikacyjna, na której aplikacja będzie pracować, będzie przechowywana w postaci grafu o etykietowanych wierzchołkach połączonych skierowanymi, ważonymi krawędziami. W czasie działania programu stale dostępny będzie podgląd obecnego stanu sieci. Graf tworzony będzie przez użytkownika za pomocą interfejsu graficznego. Program udostępni następujące operacje na sieci:
\begin{itemize}
\item dodanie nowego wierzchołka z etykietą,
\item edycja położenia lub etykiety istniejącego już wierzchołka,
\item usunięcie wierzchołka z sieci,
\item połączenie dwóch wierzchołków krawędzią,
\item usunięcie istniejącej krawędzi.
\end{itemize}
Edycja grafu będzie możliwa w każdym momencie działania programu. Aplikacja będzie przechowywać stan sieci pomiędzy uruchomieniami poprzez zapisywanie danych w odpowiednich plikach.

Aplikacja umożliwi również wykonywanie zapytań o najkrótszą ścieżkę z wierzchołka $A$ do wierzchołka $B$. W odpowiedzi na takie zapytanie program zaznaczy poszukiwaną ścieżkę osobnym kolorem na grafie i wyświetli podsumowanie trasy zawierające m.in. łączną długość ścieżki i wskazówki nawigacyjne ``krok po kroku''.

\end{document}